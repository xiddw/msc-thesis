%********************************************************************
% Appendix
%*******************************************************
% If problems with the headers: get headings in appendix etc. right
%\markboth{\spacedlowsmallcaps{Appendix}}{\spacedlowsmallcaps{Appendix}}
\chapter{Appendix Test}
Lorem ipsum at nusquam appellantur his, ut eos erant homero
concludaturque. Albucius appellantur deterruisset id eam, vivendum
partiendo dissentiet ei ius. Vis melius facilisis ea, sea id convenire
referrentur, takimata adolescens ex duo. Ei harum argumentum per. Eam
vidit exerci appetere ad, ut vel zzril intellegam interpretaris.

\begin{comment}
Entonces, es importante recordar algunas propiedades de una cadena de Márkov, que ennumera Jordan (paper 2007) y que se pueden deducir usando el teorema de separación-d.
\begin{align}
\begin{split}
  p(\mb{X} \,|\, z_n) =~ &p(x_1, ..., x_n \,|\, z_n) 
  \\ &p(x_{n+1}, ..., x_N \,|\, z_n)
\end{split} \label{eqn:3-10} \\
p(x_1, ..., x_{n-1} \,|\, x_n, z_n) =~ 
  &p(x_1, ..., x_{n-1} \,|\, z_n) 
\label{eqn:3-11} \\
p(x_1, ..., x_{n-1} \,|\, z_{n-1}, z_n) =~ 
  &p(x_1, ..., x_{n-1} \,|\, z_{n-1}) 
\label{eqn:3-12} \\
p(x_{n+1}, ..., x_N \,|\, z_n, z_{n+1}) =~ 
  &p(x_{n+1}, ..., x_N \,|\, z_{n+1}) 
\label{eqn:3-13} \\
p(x_{n+2}, ..., x_N \,|\, z_{n+1}, x_{n+1}) =~ 
  &p(x_{n+2}, ..., x_N \,|\, z_{n+1}11 
\label{eqn:3-14} \\ 
\begin{split}
  p(\mb{X} \,|\, z_{n-1}, z_n) =~ &p(x_1, ..., x_{n-1} \,|\, z_{n-1}) 
  \\ &p(x_n \,|\, z_n) 
  \\ &p(x_{n+1}, ..., x_N | z_n)
\end{split} \label{eqn:3-15} \\
  p(x_{N+1} \,|\, \mb{X} ,z_{N+1}) =~ &p(x_{N+1} \,|\, z_{N+1}) 
\label{eqn:3-16} \\
p(z_{N+1} \,|\, z_N, \mb{X}) =~ &p(z_{N+1} \,|\, z_N) 
\label{eqn:3-17} \\
\end{align}
donde $\mb{X} = \lbrce x_1, ..., x_N \rbrace$. 
\end{comment}

Errem omnium ea per, pro congue populo ornatus cu, ex qui dicant
nemore melius. No pri diam iriure euismod. Graecis eleifend
appellantur quo id. Id corpora inimicus nam, facer nonummy ne pro,
kasd repudiandae ei mei. Mea menandri mediocrem dissentiet cu, ex
nominati imperdiet nec, sea odio duis vocent ei. Tempor everti
appareat cu ius, ridens audiam an qui, aliquid admodum conceptam ne
qui. Vis ea melius nostrum, mel alienum euripidis eu.

\section{Appendix Section Test}
Ei choro aeterno antiopam mea, labitur bonorum pri no. His no decore
nemore graecis. In eos meis nominavi, liber soluta vim cu. Sea commune
suavitate interpretaris eu, vix eu libris efficiantur.

\graffito{More dummy text.}
Nulla fastidii ea ius, exerci suscipit instructior te nam, in ullum
postulant quo. Congue quaestio philosophia his at, sea odio autem
vulputate ex. Cu usu mucius iisque voluptua. Sit maiorum propriae at,
ea cum primis intellegat. Hinc cotidieque reprehendunt eu nec. Autem
timeam deleniti usu id, in nec nibh altera.

\section{Another Appendix Section Test}
Equidem detraxit cu nam, vix eu delenit periculis. Eos ut vero
constituto, no vidit propriae complectitur sea. Diceret nonummy in
has, no qui eligendi recteque consetetur. Mel eu dictas suscipiantur,
et sed placerat oporteat. At ipsum electram mei, ad aeque atomorum
mea.

\begin{table}
    \myfloatalign
  \begin{tabularx}{\textwidth}{Xll} \toprule
    \tableheadline{labitur bonorum pri no} & \tableheadline{que vista}
    & \tableheadline{human} \\ \midrule
    fastidii ea ius & germano &  demonstratea \\
    suscipit instructior & titulo & personas \\
    %postulant quo & westeuropee & sanctificatec \\
    \midrule
    quaestio philosophia & facto & demonstrated \\
    %autem vulputate ex & parola & romanic \\
    %usu mucius iisque & studio & sanctificatef \\
    \bottomrule
  \end{tabularx}
  \caption[Autem usu id]{Autem usu id.}
  \label{tab:moreexample}
\end{table}

Ei solet nemore consectetuer nam. Ad eam porro impetus, te choro omnes
evertitur mel. Molestie conclusionemque vel at, no qui omittam
expetenda efficiendi. Eu quo nobis offendit, verterem scriptorem ne
vix.

  
\begin{lstlisting}[float,caption=A floating example]
for i:=maxint to 0 do
begin
{ do nothing }
end;
\end{lstlisting}