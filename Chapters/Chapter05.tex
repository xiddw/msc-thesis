%!TEX root = ../Base.tex

\chapter{Experimentos y resultados}\label{ch:chap5}

En los capítulos anteriores se ha descrito los diferentes algoritmos que se utilizarán para realizar la tarea de \sd, y que en esta sección se emplearán de acuerdo al marco experimental que se describe a continuación.

Inicialmete, las pruebas consistieron en usar los algoritmos presentados para selección de modelo usando datos que fueron generados aleatoriamiente a partir de los parámetros de un HMM inicial; para tener una idea general de su desempeño individual.

Para estas primeras pruebas, se simuló una cadena de Márkov oculta con en base en parámetros fijos, con lo que se generó tanto una secuencia de datos observados, como los supuestos datos o variables ocultas que forman la cadena de Márkov. Se utilizó muestreo ancestral para la simulación de estos datos.

(------Agregar algoritmo de muestreo ancestral------)

Para un caso en específico, se tiene lo siguiente. Realizando la inferencia de parámetros del HMM, se obtienen los siguientes resultados:

El primer algoritmo que se prueba, es el de selección de modelo usando un BIC.

Como ya se comentó, se usará una variante de BIC en donde se incorpora un término de regularización $\lambda$ para que correspondan en órdenes de magnitud tanto la log-verosimilitud del modelo encontrado como su penalización respectiva.

El problema inmediato que se presenta, es cómo realizar la selección del parámetro de regularización $\lambda$ que penalice de forma correcta la verosimilitud para los diferentes modelos propuestos. Si $\lambda$ es demasiado pequeño, entonces la penalización realmente no tendrá efecto y dado el sobreajuste que se presenta al usar modelos más complejos, se preferirán siempre los modelos con más parámetros. Por otro lado, si al escoger $\lambda$ se da demasiado peso al término de penalización, entonces siempre se preferirán los modelos más sencillos.

Para encontrar el valor de $\lambda$ adecuado, se puede entonces formar una superficie con las diferentes curvas de selección BIC de acuerdo a cómo varía $\lambda$.

