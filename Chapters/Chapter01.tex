%!TEX root = ../Base.tex

%************************************************
\chapter{Introducción}\label{ch:chap1}
%************************************************

\epigraphhead[70]{
\epigraph{Essentially, all models are wrong, but some are useful.}
         {George E. P. Box}}
         
En los últimos años la tarea de \SD~ se ha vuelto una parte importante de diferentes procesos que se realizan con las grabaciones de audio, tales como la identificación y navegación por segmentos en específico, así como la búsqueda y recuperación de información en grandes volúmenes de secuencias de audio.

La investigación desarrollada referente \sd~ se ha guiado principalmente de acuerdo a la financiación existente para proyectos específicos. Hasta principios de la década de 1990, el trabajo de investigación se concentraba en trabajar con grabaciones telefónicas. Principalmente se usaba para segmentar la conversación, así como etapa de pre-procesamiento para luego realizar reconocimiento y/o transcripción del habla.

Para la década del 2000, las aplicaciones fueron cambiando, así como aumentando la capacidad disponible de almacenamiento; por lo que creció el interés de mantener un registro de forma automática de noticieros televisivos  y transmisiones de radio a lo largo de todo el planeta. Entre la información más útil que se registraba de las grabaciones era la transcripción del diálogo, meta-etiquetas referentes al contenido así como la segmentación y le orden de las personas que intervienen en la grabación.

A principios del año 2002, empezaron a surgir varios proyectos de investigación cuya principal intención era mejorar la comunicación interpersonal, y en especial la que ocurre a larga distancia y de forma multimodal. La investigación y desarrollo se enfocó en extracción del contenido y su etiquetación de acuerdo a las personas que participan, ya sea para mantener un archivo histórico, o para fácil disposición de personas interesadas en su contenido.

De acuerdo a todo este trabajo desarrollado, se puede distinguir que hay dos grandes grupos de aplicaciones para \sd: las que usan 

\section{Trabajo previo}
\label{sec:previo}
