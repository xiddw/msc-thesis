%!TEX root = ../Base.tex

%************************************************
\chapter{Introducción}\label{ch:chap1}
%************************************************

%\epigraphhead[70]{
%\epigraph{Essentially, all models are wrong, but some are useful.}
%         {George E. P. Box}}
         
En los últimos años la tarea de \SD~ se ha vuelto una parte importante de diferentes procesos que se realizan con las grabaciones de audio, tales como la identificación y navegación por segmentos en específico, así como la búsqueda y recuperación de información en grandes volúmenes de secuencias de audio.

La investigación desarrollada referente a \sd~ se ha guiado principalmente de acuerdo a la financiación existente para proyectos específicos. Hasta principios de la década de 1990, el trabajo de investigación se concentraba en trabajar con grabaciones telefónicas. Principalmente se usaba para segmentar la conversación, así como etapa de pre-procesamiento para luego realizar reconocimiento y/o transcripción del habla.

Para la década del 2000, las aplicaciones fueron cambiando, así como aumentando la capacidad disponible de almacenamiento; por lo que creció el interés de mantener un registro de forma automática de noticieros televisivos  y transmisiones de radio a lo largo de todo el planeta. Entre la información más útil que se registraba de las grabaciones, era la transcripción del diálogo, meta-etiquetas referentes al contenido así como la segmentación y le orden de las personas que intervienen en la grabación.

A principios del año 2002, empezaron a surgir varios proyectos de investigación cuya principal intención era mejorar la comunicación interpersonal, y en especial la que ocurre a larga distancia y de forma multimodal. La investigación y desarrollo se enfocó en extracción del contenido y su etiquetación de acuerdo a las personas que participan, ya sea para mantener un archivo histórico, o para fácil disposición de personas interesadas en su contenido.

Debido a la creciente investigación en estos campos, el Instituto Nacional para los Estándares y la Tecnología de los Estados Unidos (NIST por sus siglas en inglés), ha organizado un sistema de oficial de evaluaciones que permita  unificar así como dirigir el esfuerzo de los investigadores que trabajan en esta área; al tener una manera precisa de comparar las diferentes metodologías en las que trabajan. Primero, en el 2002 las pruebas consistían en grabaciones correspondientes a noticieros, y para el 2004 se empezaron a incluir pruebas con grabaciones de reuniones, que resultaban ser la principal aplicación en esos años.

Una gran diferencia entre ambos tipos de entornos, es que mientras en un noticiero se suelen tener preparados los diálogos que se dirán e incluso se leen las noticias; en el caso de las reuniones la conversación es más espontánea y puede suceder que dos personas hablen al mismo tiempo. 

Otra característica diferente, es que mientras en un noticiero cada participante suele tener un micrófono de solapa o hay micrófonos ambientales, éstos suelen ser de calidad, por lo que no el audio de la grabación es mucho mejor y no hay mucho ruido presente en la señal. 

Por otro lado, para el caso de las reuniones, el ambiente no suele ser tan controlado, y puede haber un mayor ruido ambiental; además de que tanto al disposición de los micrófonos así como su calidad no siempre son la mejor, agregando interferencia o redundancia innecesaria a la señal.

Por eso, se considera el caso de reuniones como el escenario completo para las tareas involucradas con reconocimiento del habla; por la complejidad y los diferentes problemas que se pueden presentar.

En este trabajo de tesis se abordará el problema usando un escenario similar al del noticiero, en donde no hay ruido ambiental y los diálogos son más o menos continuos, además de que los participantes no se interrumpen al hablar.

\section{Trabajo previo}
\label{sec:previo}

De acuerdo a todo este trabajo desarrollado, se puede distinguir que hay dos grandes enfoques que se usan para \sd: \textit{bottom-up} y \textit{top-down}

