\pagestyle{empty}
\hfill \vfill

\section*{\hspace*{-40pt}Colofón}
\bigskip
%\pdfbookmark[0]{Colofón}{Colofón}

\hspace*{120pt}
%\hspace{-60pt}
\parbox{230pt}{\lettrine[lines=3]{\textcolor{Maroon}{E}}{ste documento} fue escrito utilizando \LaTeX, originalmente desarrollado por Leslie Lamport; y basado en \TeX~ de Donald Knuth.}

\vspace{20pt}
\hspace*{30pt}
%\hspace{-60pt}
\parbox{320pt}{\lettrine[lines=3, slope=1pt]{\textcolor{Maroon}{A}}{demás, se utilizó la plantilla}
\texttt{classicthesis}, desarrollada por Andr\'e Miede, cuyo estilo fue inspirado por el libro de Robert Bringhurs ``\emph{The Elements of Typographic Style}''.
\\~\\
\texttt{classicthesis} está disponible tanto para \LaTeX\ como para \mLyX: \\~\\
\url{http://code.google.com/p/classicthesis/}
}

\vspace{20pt} \hspace{-60pt}
\parbox{410pt}{\lettrine[lines=3]{\textcolor{Maroon}{L}}{a mayoría de las figuras} de esta tesis usan una paleta de colores diseñada para personas con deficiencia en la percepción del color. Más información sobre este tipo de paletas puede ser encontrado en el articulo de M. Geissbuehler and T. Lasser.
\\~\\
\href{http://lob.epfl.ch/page-89396-en.html}{
``\emph{How to display data by color schemes compatible with red-green color perception deficiencies}''.}
}

\bigskip

\noindent\finalVersionString \\~\\
Hermann Zapf's \emph{Palatino} and \emph{Euler} type faces (Type~1 PostScript fonts \emph{URW Palladio L} and \emph{FPL}) are used. The ``typewriter'' text is typeset in \emph{Bera Mono}, originally developed by Bitstream, Inc. as ``Bitstream Vera''. (Type~1 PostScript fonts were made available by Malte Rosenau and Ulrich Dirr.)

%\paragraph{note:} The custom size of the textblock was calculated
%using the directions given by Mr. Bringhurst (pages 26--29 and
%175/176). 10~pt Palatino needs  133.21~pt for the string
%``abcdefghijklmnopqrstuvwxyz''. This yields a good line length between
%24--26~pc (288--312~pt). Using a ``\emph{double square textblock}''
%with a 1:2 ratio this results in a textblock of 312:624~pt (which
%includes the headline in this design). A good alternative would be the
%``\emph{golden section textblock}'' with a ratio of 1:1.62, here
%312:505.44~pt. For comparison, \texttt{DIV9} of the \texttt{typearea}
%package results in a line length of 389~pt (32.4~pc), which is by far
%too long. However, this information will only be of interest for
%hardcore pseudo-typographers like me.%
%
%To make your own calculations, use the following commands and look up
%the corresponding lengths in the book:
%\begin{verbatim}
%    \settowidth{\abcd}{abcdefghijklmnopqrstuvwxyz}
%    \the\abcd\ % prints the value of the length
%\end{verbatim}
%Please see the file \texttt{classicthesis.sty} for some precalculated 
%values for Palatino and Minion.
%
%    \settowidth{\abcd}{abcdefghijklmnopqrstuvwxyz}
%    \the\abcd\ % prints the value of the length
